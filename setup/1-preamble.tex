\documentclass[12pt,twoside,a4paper,openright]{report}
% \documentclass[12pt,a4paper]{report}
%%%%%%%%%%%%%%%%%%%%%%%%%%%%%%%%%%%%%%%%%%%%%%%%
% Title and other reusable elements
% http://en.wikibooks.org/wiki/LaTeX/Internationalization
%%%%%%%%%%%%%%%%%%%%%%%%%%%%%%%%%%%%%%%%%%%%%%%%
% choose lanuage
\newcommand{\mylanguage}{danish} % danish or english
\newcommand{\mytitle}{Attiude Control and Magnetorquer}
\title{\mytitle}
\newcommand{\mysubtitle}{Monitorering af fejl i trailerkredsløb og advarsel af bruger}
\newcommand{\myprojecttype}{Semester Projekt} % Semester Project % Master's Project % Bachelor Project
\newcommand{\myeducation}{ESD} % three letter acronym
\newcommand{\mysemester}{2} % in numbers
\newcommand{\mygroup}{214} % in numbers
\newcommand{\mytheme}{Satellite}
% find a good way to write the names and emails here
% when using the author names it need to be able to be formated

%%%%%%%%%%%%%%%%%%%%%%%%%%%%%%%%%%%%%%%%%%%%%%%%
% Language
% http://en.wikibooks.org/wiki/LaTeX/Internationalization
%%%%%%%%%%%%%%%%%%%%%%%%%%%%%%%%%%%%%%%%%%%%%%%%

% used to detect the language
\usepackage{iflang}

% Make latex understand and use the typographic
% rules of the language used in the document.
\usepackage[english,danish]{babel}

%%%%%%%%%%%%%%%%%%%%%%%%%%%%%%%%%%%%%%%%%%%%%%%%
% Graphics and Tables
% http://en.wikibooks.org/wiki/LaTeX/Importing_Graphics
% http://en.wikibooks.org/wiki/LaTeX/Tables
% http://en.wikibooks.org/wiki/LaTeX/Colors
%%%%%%%%%%%%%%%%%%%%%%%%%%%%%%%%%%%%%%%%%%%%%%%%
% load a colour package
\usepackage[table]{xcolor}
\definecolor{aaublue}{RGB}{34, 35, 78}% dark blue

% The standard graphics inclusion package
\usepackage{graphicx}
\usepackage[inkscapelatex=false]{svg}


% Use images with a correct copyright attributions
\usepackage{caption, copyrightbox, ccicons}

% Set up how figure and table captions are displayed
\usepackage{caption}
\captionsetup{%
	font=footnotesize,% set font size to footnotesize
	labelfont=bf % bold label (e.g., Figure 3.2) font
}
\usepackage{subcaption}

% new floating environments
\usepackage{newfloat}
% Lists
% \newlistof{figlist}{los}{List of Lists}
\DeclareFloatingEnvironment[
	fileext=los,
	listname={List of Lists},
	name=List,
	placement=tbhp,
	within=section,
]{figlist}
%Set names for the English language
\addto\captionsenglish{%
	\renewcommand\figlistname{List}%
	\renewcommand\listfiglistname{List of Lists}%
}
%Set names for the Danish language
\addto\captionsdanish{%
	\renewcommand\figlistname{Liste}%
	\renewcommand\listfiglistname{Lister}%
}

% Make the standard latex tables look so much better
\usepackage{array,booktabs}
% Used to diffine contant with tables
\usepackage{tabularx} % http://ctan.org/pkg/tabularx
% \usepackage{makecell}
\usepackage{multirow}
\usepackage[l3]{csvsimple}
\usepackage{pgf-pie}
\usepackage{longtable}

% Enable the use of frames around, e.g., theorems
% The framed package is used in the example environment
\usepackage{framed}

% Adds support for full page background picture
\usepackage[contents={},color=gray]{background}
%\usepackage[contents=draft,color=gray]{background}

% https://www.overleaf.com/learn/latex/Code_Highlighting_with_minted
\usepackage{minted}
\usemintedstyle{xcode}
% inline code
\definecolor{light-gray}{gray}{0.95}
\newcommand{\codeCpp}[1]{\mintinline[breaklines,breakafter=_]{c++}|#1|}


%%%%%%%%%%%%%%%%%%%%%%%%%%%%%%%%%%%%%%%%%%%%%%%%
% Mathematics
% http://en.wikibooks.org/wiki/LaTeX/Mathematics
%%%%%%%%%%%%%%%%%%%%%%%%%%%%%%%%%%%%%%%%%%%%%%%%
% Defines new environments such as equation,
% align and split
\usepackage{amsmath}
% Adds new math symbols
\usepackage{amssymb}
% Use theorems in your document
% The ntheorem package is also used for the example environment
% When using thmmarks, amsmath must be an option as well. Otherwise
% \eqref doesn't work anymore.
\usepackage[framed,amsmath,thmmarks]{ntheorem}
% To use SI units in math
\usepackage{siunitx}
% \usepackage{icomma} % %comma seperator
\usepackage{derivative}

%%%%%%%%%%%%%%%%%%%%%%%%%%%%%%%%%%%%%%%%%%%%%%%%
% Tikz and PGF 
% https://www.overleaf.com/learn/latex/TikZ_package
% https://www.overleaf.com/learn/latex/CircuiTikz_package
% https://ctan.org/pkg/gnuplottex?lang=en
%%%%%%%%%%%%%%%%%%%%%%%%%%%%%%%%%%%%%%%%%%%%%%%%
\usepackage{tikz}
\usetikzlibrary{calc,patterns,angles,quotes,decorations.pathmorphing,decorations.text,math,decorations.pathreplacing,decorations.markings,automata,arrows.meta,positioning,external}
\usepackage[european]{circuitikz}
\usepackage{tikzsymbols}
\usepackage{pgfplots}
\pgfplotsset{compat=1.18}
% \usepackage[siunitx]{gnuplottex}
\usepackage{pgfplots}
\usepackage[most]{tcolorbox}
\usepackage{xcolor}


%%%%%%%%%%%%%%%%%%%%%%%%%%%%%%%%%%%%%%%%%%%%%%%%
% Page Layout
% http://en.wikibooks.org/wiki/LaTeX/Page_Layout
%%%%%%%%%%%%%%%%%%%%%%%%%%%%%%%%%%%%%%%%%%%%%%%%
% Change margins, papersize, etc of the document
\usepackage[
	inner=28mm,% left margin on an odd page
	outer=41mm,% right margin on an odd page
	% top=-4pt,
]{geometry}

% Modify how \chapter, \section, etc. look
% The titlesec package is very configureable
\usepackage{titlesec}
\titleformat{\chapter}[display]{\normalfont\huge\bfseries}{\chaptertitlename\ \thechapter}{20pt}{\Huge}
\titleformat*{\section}{\normalfont\Large\bfseries}
\titleformat*{\subsection}{\normalfont\large\bfseries}
\titleformat*{\subsubsection}{\normalfont\normalsize\bfseries}
%\titleformat*{\paragraph}{\normalfont\normalsize\bfseries}
%\titleformat*{\subparagraph}{\normalfont\normalsize\bfseries}

% Clear empty pages between chapters
\let\origdoublepage\cleardoublepage
\newcommand{\clearemptydoublepage}{%
	\clearpage
	{\pagestyle{empty}\origdoublepage}%
}
\let\cleardoublepage\clearemptydoublepage

% Change the headers and footers
\usepackage{fancyhdr}
\fancypagestyle{plain}{%
	\fancyhf{} %delete everything
	\renewcommand{\headrulewidth}{0pt} %remove the horizontal line in the header
	\fancyfoot[LO,RE]{\thepage} %page number on all pages
	\raggedbottom
}
\fancypagestyle{fancy}{%
	\fancyhf{} %delete everything
	\renewcommand{\headrulewidth}{0pt} %remove the horizontal line in the header
	\fancyhead[LO,RE]{\large \mytitle} % Name of project
	\fancyhead[LE,RO]{\myeducation\mysemester\ G\mygroup} % Groupe number on all pages
	\fancyfoot[LO,RE]{\thepage \ af \pageref{LastPage}} %page number on all pages
	% Do not stretch the content of a page. Instead,
	% insert white space at the bottom of the page
	\raggedbottom
}
\pagestyle{fancy}
\fancypagestyle{chapter}{%
	% \pagestyle{fancy}
	\fancyhf{} %delete everything
	\renewcommand{\headrulewidth}{0pt} %remove the horizontal line in the header
	\fancyfoot[LO,RE]{\thepage \ af \pageref{LastPage}} %page number on all pages
	% Do not stretch the content of a page. Instead,
	% insert white space at the bottom of the page
	% \headheight{18px}

	\raggedbottom
}

\usepackage{titlesec}
% Chapter style
\titleformat
{\chapter}% command
[display]% shape
{
	\thispagestyle{chapter}
	\Huge
	\bfseries
}% format
{\vspace{-2cm}
	\filleft\hspace*{-60pt}{\Huge\colorbox{white}{\parbox[c][2cm][c]{2cm}{\fontsize{2cm}{1cm}\selectfont\centering\thechapter}}}
}% label
{0.5cm}% sep
{
	\vspace{0.5cm}
	\filleft
}% before-code
[
]% after-code

% Enable arithmetics with length. Useful when
% typesetting the layout.
\usepackage{calc}

% Used to write dommy text
\usepackage{lipsum}

% Used to make begin comment
\usepackage{comment}

%%%%%%%%%%%%%%%%%%%%%%%%%%%%%%%%%%%%%%%%%%%%%%%%
% Bibliography
% http://en.wikibooks.org/wiki/LaTeX/Bibliography_Management
%%%%%%%%%%%%%%%%%%%%%%%%%%%%%%%%%%%%%%%%%%%%%%%%
\usepackage[backend=biber,
	bibencoding=utf8,
	style=numeric-comp,
	sorting=none
]{biblatex}
\addbibresource{bib/main.bib}


%%%%%%%%%%%%%%%%%%%%%%%%%%%%%%%%%%%%%%%%%%%%%%%%
% Misc
%%%%%%%%%%%%%%%%%%%%%%%%%%%%%%%%%%%%%%%%%%%%%%%%
% Add bibliography and index to the table of
% contents
\usepackage[nottoc]{tocbibind}
% Add the command \pageref{LastPage} which refers to the
% page number of the last page
\usepackage{lastpage}
% Add todo notes in the margin of the document
\usepackage[
	%  disable, %turn off todonotes
	colorinlistoftodos, %enable a coloured square in the list of todos
	textwidth=\marginparwidth, %set the width of the todonotes
	textsize=scriptsize, %size of the text in the todonotes
]{todonotes}

% Suppress warnings
\usepackage{silence}
\WarningsOff[everypage]% Suppress warnings related to package everypage

% Time 
\usepackage{datetime2}
\DTMsetdatestyle{iso}

%%%%%%%%%%%%%%%%%%%%%%%%%%%%%%%%%%%%%%%%%%%%%%%%
% Hyperlinks
% http://en.wikibooks.org/wiki/LaTeX/Hyperlinks
%%%%%%%%%%%%%%%%%%%%%%%%%%%%%%%%%%%%%%%%%%%%%%%%
% Enable hyperlinks and insert info into the pdf
% file. Hypperref should be loaded as one of the
% last packages
\usepackage{hyperref}
\hypersetup{%
	% pdfpagelabels=true,%
	plainpages=false,%
	pdfauthor={Author(s)},%
	pdftitle={Title},%
	pdfsubject={Subject},%
	pdfborder=0 0 0,
	bookmarksnumbered=true,%
	colorlinks=false,%
	citecolor=black,%
	filecolor=black,%
	linkcolor=black,% you should probably change this to black before printing
	urlcolor=black,%
	pdfstartview=FitH%
}
% linebreaks in \url
\usepackage[]{xurl}



\usepackage{float}
\usepackage{enumitem}
\usepackage[danish=guillemets]{csquotes}

%%%%%%%%%%%%%%%%%%%%%%%%%%%%%%%%%%%%%%%%%%%%%%%%
% Project Management
% https://www.overleaf.com/learn/latex/Management_in_a_large_project
%%%%%%%%%%%%%%%%%%%%%%%%%%%%%%%%%%%%%%%%%%%%%%%%
% \usepackage{import}
\usepackage{subfiles} % Best loaded last in the preamble

%%%%%%%%%%%%%%%%%%%%%%%%%%%%%%%%%%%%%%%%%%%%%%%%
% Encoding and Fonts
% http://en.wikibooks.org/wiki/LaTeX/Internationalization
%%%%%%%%%%%%%%%%%%%%%%%%%%%%%%%%%%%%%%%%%%%%%%%%

% Choose the font encoding
\usepackage{fontspec}
\usepackage[
math-style=ISO,
bold-style=ISO,
mathrm=sym,
mathup=sym,
mathit=sym,
mathsf=sym,
mathbf=sym,
mathtt=sym,
]{unicode-math}
\setmainfont{LibertinusSerif}[
	Path = \subfix{setup/fonts/Libertinus/static/OTF/},
	Scale=1,
	Contextuals = Alternate,
	Extension = .otf,
	UprightFont =*-Regular,
	BoldFont = *-Bold,
	ItalicFont = *-Italic,
	BoldItalicFont = *-BoldItalic,
]
\setmathfont{LibertinusMath-Regular}[ % for math symbols, can be any other OpenType math font
	Path = \subfix{setup/fonts/Libertinus/static/OTF/},
	Scale=1,
	% Ligatures=TeX,
	Extension = .otf,
]

\setmonofont{JetBrainsMonoNerdFont}[
	Path = \subfix{setup/fonts/JetBrainsMono/},
	Scale=0.85,
	Contextuals = Alternate,
	Extension = .ttf,
	UprightFont =*-Regular,
	BoldFont = *-Bold,
	ItalicFont = *-Italic,
	BoldItalicFont = *-BoldItalic,
]

\usepackage{emoji}
\setemojifont{Noto Color Emoji}


