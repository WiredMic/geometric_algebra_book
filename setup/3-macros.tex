% see, e.g., http://en.wikibooks.org/wiki/LaTeX/Customizing_LaTeX#New_commands
% for more information on how to create macros

%%%%%%%%%%%%%%%%%%%%%%%%%%%%%%%%%%%%%%%%%%%%%%%%
% Macros for the titlepage
%%%%%%%%%%%%%%%%%%%%%%%%%%%%%%%%%%%%%%%%%%%%%%%%
%Creates the aau titlepage
\newcommand{\aautitlepage}[3]{%
  {
    %set up various length
    \ifx\titlepageleftcolumnwidth\undefined
    \newlength{\titlepageleftcolumnwidth}
    \newlength{\titlepagerightcolumnwidth}
    \fi
    \setlength{\titlepageleftcolumnwidth}{0.5\textwidth-\tabcolsep}
    \setlength{\titlepagerightcolumnwidth}{\textwidth-2\tabcolsep-\titlepageleftcolumnwidth}
    %create title page
    \thispagestyle{empty}
    \noindent%
    \begin{tabular}{@{}ll@{}}
      \parbox{\titlepageleftcolumnwidth}{
        \iflanguage{danish}{%
          \includegraphics[width=\titlepageleftcolumnwidth]{AAUgraphics/aau_logo_da}
        }{%
          \includegraphics[width=\titlepageleftcolumnwidth]{AAUgraphics/aau_logo_en}
        }
      } &
      \parbox{\titlepagerightcolumnwidth}{\raggedleft\sf\small
        #2
      }\bigskip\\
      #1 &
      \parbox[t]{\titlepagerightcolumnwidth}{%
        \textbf{Abstract:}\bigskip\par
        \fbox{\parbox{\titlepagerightcolumnwidth-2\fboxsep-2\fboxrule}{%
            #3
        }}
      }\\
    \end{tabular}
    \vfill
    \iflanguage{danish}{%
      \noindent{\footnotesize\emph{Rapportens indhold er frit
          tilgængeligt, men offentliggørelse (med kildeangivelse) må kun
      ske efter aftale med forfatterne.}}
    }{%
      \noindent{\footnotesize\emph{The content of this report is
          freely available, but publication (with reference) may only be
      pursued due to agreement with the author.}}
    }
    \cleardoublepage
  }
}


\newcommand{\projectinfo}[8]{%
  \parbox[t]{\titlepageleftcolumnwidth}{
    \iflanguage{danish}{%
    \textbf{Titel:}}{
    \textbf{Title:}}\\ #1\bigskip\par

   
    \iflanguage{danish}{%
    \textbf{Tema:}}{
    \textbf{Theme:}}\\ #2\bigskip\par

    
    \iflanguage{danish}{%
    \textbf{Projektperiode:}}{
    \textbf{Project Period:}}\\ #3\bigskip\par

    
    \iflanguage{danish}{%
    \textbf{Projektgruppe:}}{
    \textbf{Project Group:}}\\ #4\bigskip\par

    
    \iflanguage{danish}{%
    \textbf{Deltager(e):}}{
    \textbf{Participant(s):}}\\ #5\bigskip\par

    
    \iflanguage{danish}{%
    \textbf{Vejleder(e):}}{
    \textbf{Supervisor(s):}}\\ #6\bigskip\par

    
    \iflanguage{danish}{%
    \textbf{Oplagstal:}}{
    \textbf{Copies:}} #7\bigskip\par

    
    \iflanguage{danish}{%
    \textbf{Sidetal:}}{
    \textbf{Page Numbers:}} \pageref{LastPage}\bigskip\par

    
    \iflanguage{danish}{%
    \textbf{Afleveringsdato:}}{
    \textbf{Date of Completion:}}\\ #8
  }
}


%Create english project info
\newcommand{\englishprojectinfo}[8]{%
  \parbox[t]{\titlepageleftcolumnwidth}{
    \textbf{Title:}\\ #1\bigskip\par
    \textbf{Theme:}\\ #2\bigskip\par
    \textbf{Project Period:}\\ #3\bigskip\par
    \textbf{Project Group:}\\ #4\bigskip\par
    \textbf{Participant(s):}\\ #5\bigskip\par
    \textbf{Supervisor(s):}\\ #6\bigskip\par
    \textbf{Copies:} #7\bigskip\par
    \textbf{Page Numbers:} \pageref{LastPage}\bigskip\par
    \textbf{Date of Completion:}\\ #8
  }
}

%Create danish project info
\newcommand{\danishprojectinfo}[8]{%
  \parbox[t]{\titlepageleftcolumnwidth}{
    \textbf{Titel:}\\ #1\bigskip\par
    \textbf{Tema:}\\ #2\bigskip\par
    \textbf{Projektperiode:}\\ #3\bigskip\par
    \textbf{Projektgruppe:}\\ #4\bigskip\par
    \textbf{Deltager(e):}\\ #5\bigskip\par
    \textbf{Vejleder(e):}\\ #6\bigskip\par
    \textbf{Oplagstal:} #7\bigskip\par
    \textbf{Sidetal:} \pageref{LastPage}\bigskip\par
    \textbf{Afleveringsdato:}\\ #8
  }
}

%%%%%%%%%%%%%%%%%%%%%%%%%%%%%%%%%%%%%%%%%%%%%%%%
% An example environment
%%%%%%%%%%%%%%%%%%%%%%%%%%%%%%%%%%%%%%%%%%%%%%%%
\theoremheaderfont{\normalfont\bfseries}
\theorembodyfont{\normalfont}
\theoremstyle{break}
\def\theoremframecommand{{\color{gray!50}\vrule width 5pt \hspace{5pt}}}
\newshadedtheorem{exa}{Example}[chapter]
\newenvironment{example}[1]{%
  \begin{exa}[#1]
  }{%
  \end{exa}
}



